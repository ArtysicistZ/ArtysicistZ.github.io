\documentclass[
	%a4paper, % Uncomment for A4 paper size (default is US letter)
	10pt, % Default font size, can use 10pt, 11pt or 12pt
]{resume} % Use the resume class

\usepackage{ebgaramond} % Use the EB Garamond font
\usepackage[hidelinks]{hyperref} % Clickable email/URL without colored boxes
\urlstyle{same} % Keep URLs in the same font as the body text

%------------------------------------------------

\name{Yincheng Zhou} % Your name to appear at the top

% You can use the \address command up to 3 times for 3 different addresses or pieces of contact information
% Any new lines (\\) you use in the \address commands will be converted to symbols, so each address will appear as a single line.

% \address{3333 Walnut Street \\ Philadelphia, PA 19104} 

% Contact line: phone, email, GitHub, website (single line, clear separators)
\address{(267) 788-2488\;|\;\href{mailto:yzhou29@seas.upenn.edu}{yzhou29@seas.upenn.edu}\;|\;\href{https://github.com/ArtysicistZ}{github.com/ArtysicistZ}\;|\;\href{https://artysicistz.github.io}{artysicistz.github.io}} % Contact information

%----------------------------------------------------------------------------------------

\begin{document}

%----------------------------------------------------------------------------------------
%	EDUCATION SECTION
%----------------------------------------------------------------------------------------

\begin{rSection}{Education}

	\begin{rSubsection}{University of Pennsylvania}{September 2025 - Present}{Vagelos Integrated Program in Energy Research (VIPER)}{Philadelphia, PA}
		\item Dual Degree: B.S.E. in Computer \& Information Science (AI Concentration)\;|\;B.A. in Physics \& Astronomy
		\item Relevant Coursework: Big Data Analysis, Computer Systems, Data Structures \& Algorithms
		\item GPA: 4.0 / 4.0
	\end{rSubsection}

\end{rSection}

%----------------------------------------------------------------------------------------
%	INTERNSHIP EXPERIENCE SECTION
%----------------------------------------------------------------------------------------

\begin{rSection}{Experience}

	\begin{rSubsection}{Franklink, Inc.\,\,{\small\normalfont[\href{https://artysicistz.github.io/projects/franklink-demo.html}{Demo}]}}{October 2025 - Feburary 2026}{Founding Software Engineer}{Philadelphia, PA}
		
		\item Built an iMessage-based networking assistant, automating matchmaking via multi-agent orchestration.
		\item Implemented persisted agent memory and selective dispatch across 5 sub-agents; cut user-facing latency by 30\%.
		\item Designed agentic RAG with permissioned access across 8 tools (DB, email, calendar) for context-aware retrieval and recommendation.
		\item Engineered a Kafka-backed event pipeline for LLM-heavy messaging with at-least-once delivery, idempotency keys, and retry topics (backoff) + DLQ; supported 50+ concurrent sessions while keeping p99 end-to-end latency within 40\% of single-session baseline.
		\item Containerize services and design test/deploy workflows; ship on AWS ECS.

	\end{rSubsection}

%------------------------------------------------

	\begin{rSubsection}{University of Pennsylvania, GRASP Laboratory}{January 2026 - Present}{Machine Learning Researcher $|$ Advisor: Prof. Pratik Chaudhari}{Philadelphia, PA}

		\item Designed SPD, a sparse depth prediction algorithm that decodes only at queried pixels, achieving 2.9$\times$ inference speedup over dense baselines for real-time robotics.
		\item Implemented a lightweight per-query decoder (11M params) with multi-scale cross-attention, cross-query communication, and deformable sampling, enabling 72 Hz inference at 256 query points with 544*416 resolution on RTX 4060.
		\item Architected dense canvas supervision strategy to address sparse query information loss during training, enabling convergence on NYU Depth V2 without ground truth densification.

	\end{rSubsection}

\end{rSection}

%----------------------------------------------------------------------------------------
%	PROJECTS SECTION
%----------------------------------------------------------------------------------------

\begin{rSection}{Projects}

	\begin{rSubsection}{AlphaOne\,\,{
		\small
		\normalfont[\href{https://github.com/ArtysicistZ/AlphaOne.git}{GitHub}]\,
		[\href{https://artysicistz.github.io/projects/alphaone-demo.html}{Demo}]\,
		[\href{https://huggingface.co/ArtysicistZ/deberta-absa-v2}{Play with DeBERTa-ABSA-v2}]
	}}{October 2025 - Present}{Tech Stack: PyTorch, HuggingFace, React, Spring Boot, FastAPI, Celery, Redis, PostgreSQL}{}
		\item Fine-tuned DeBERTa to classify sentiment per ticker in multi-entity financial text; achieving 82.5\% accuracy (0.823 Macro F1), increasing performance by 65 percentage points from FinBERT baseline.
		\item Curated 6,287 training triples: scraped 7 subreddits, LLM-labeled via Ollama, hand-audited 4,501 pairs (615 corrections).
		\item Architected a 6-layered sentiment pipeline with idempotent content hashing for concurrent-safe batch ingestion.
		\item Engineered concurrent-safe, idempotent batch ingestion with content versioning; deployed via Docker Compose.
	\end{rSubsection}

%------------------------------------------------

	\begin{rSubsection}{NL2SQL Bot\,\,{\small\normalfont[\href{https://github.com/ArtysicistZ/NL2SQL_Bot.git}{GitHub}]\,[\href{https://artysicistz.github.io/projects/nl2sql.html}{Demo}]}}{December 2025 - January 2026}{Tech Stack: Google ADK, FastAPI, MySQL, Plotly.js}{}
		\item Built a multi-agent NL2SQL system on Google ADK converting natural language to validated SQL with interpretive visualization.
		\item Orchestrated 5 agents with session state passing, loop-guarded retry, and per-agent LLM overrides for cost-performance tuning.
		\item Enforced read-only execution via 18 blocked SQL patterns and table allowlisting; abstracted dialect rules across MySQL, PostgreSQL, and SQLite.
	\end{rSubsection}

\end{rSection}

%----------------------------------------------------------------------------------------
%	SKILLS & ACHIEVEMENTS SECTION
%----------------------------------------------------------------------------------------

\begin{rSection}{Skills \& Achievements}

  	\begin{tabular}{@{} >{\bfseries}l @{\hspace{6ex}} l @{}}
    	Languages & C++, Python, Java, JavaScript/TypeScript, SQL, Assembly, OCaml \\
    	Backend \& DevTools & FastAPI, Spring Boot, Kafka, Redis, Celery, Docker, AWS, Git \\
    	ML \& Data & PyTorch, CUDA, spaCy, Pandas, LangGraph, Google ADK \\
		Achievements & Team China, International Science and Engineering Fair (ISEF)\\
		& Silver Medal, Team China, International Young Physicists' Tournament (IYPT) \\
		& Second Prize, China Mathematics Olympiad\\
		& Second Prize, China Physics Olympiad, Second Prize\\
		& Champion, Start-up in a Weekend Hackathon, Photon Hybrid Intelligence Track

  	\end{tabular}

\end{rSection}

%----------------------------------------------------------------------------------------

\end{document}
